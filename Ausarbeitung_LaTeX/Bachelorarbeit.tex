%%%%%%%%%%%%%%%%%%%%%%%%%%%%%%%%%%%%%%%%%%%%%%%%%
%------ LaTeX-Template
%%%%%%%%%%%%%%%%%%%%%%%%%%%%%%%%%%%%%%%%%%%%%%%%%

%---- Header (mit Formateinstellugen) laden, Inputencoding prüfen ------

\input{header}

%\usepackage[applemac]{inputenc} % Inputencoding f�r Mac
%\usepackage[latin1]{inputenc} % Inputencoding f�r PC/Win
\usepackage[utf8]{inputenc} % Inputencoding, universell
%\usepackage[utf8x]{inputenc} % Inputencoding, universell


%------------------------ Titelblatt-Layout laden ----------------------------------

%%%%%%%%%%%%%%%%%%%%%%%%%%%%%%%%%%%%%%%%%%%%%%%%%
%------ LaTeX-Titelblatt
%------ Deklarationen fuer die Titelseite
%%%%%%%%%%%%%%%%%%%%%%%%%%%%%%%%%%%%%%%%%%%%%%%%%

\title{\titel\\[2ex]
\LARGE Bachelor-Thesis\\
\large zur Erlangung des akademischen Grades B.Sc.\\[1.5ex]
\LARGE \vorname~\nachname\\[0.5ex] 
\large \matrikelnummer
}

\author{\unitlength1mm
\large\raisebox{-1ex}{\includegraphics[width=4em]{Bilder/HAW_wuerfel}}\hspace{1ex}
\parbox[b]{11.2cm}{\sffamily\large%
Hochschule für Angewandte Wissenschaften Hamburg\\[-0.2ex]
Fakultät Design, Medien und Information\\[-0.2ex]
Department Medientechnik
}\\[6ex]
\sffamily\large Erstprüfer: \erstpruef\\[0.5ex]
\sffamily\large Zweitprüfer: \zweitpruef}

%%%%%%%%%%%%%%%%%%%%%%%%%%%%%%%%%%%%%%%%%%%%%%%%%

%---------------------------- Titeldefinitionen --------------------------------------

\newcommand{\vorname}{Sebastian}
\newcommand{\nachname}{Bohn}
\newcommand{\matrikelnummer}{2036605}

\newcommand{\titel}{Analyse und Evaluierung von plattformübergreifenden Spiel-Engines und Frameworks,anhand der Implementierung einer mobilen Beispielapplikation}

\newcommand{\erstpruef}{Prof. Dr. Edmund Weitz}
\newcommand{\zweitpruef}{Prof. Dr. Andreas Plaß}

\date{vorläufige Fassung vom \today}   % Vorab-Version 
%\date{\sffamily Hamburg, DD. MM. YYYY}  % Abgabedatum!

%----------------------------- ANFANG --------------------------------------

\begin{document}
\selectlanguage{ngerman}
% Titelseite erzeugen
\maketitle
% Inhaltsverzeichnis erzeugen          
\tableofcontents
% Seitenumbruch
\clearpage

%----------------------------INHALT---------------------------------------

%---------------------------ABSTRACT------------------------------------

\thispagestyle{empty}
\selectlanguage{english}
\section*{\centering\abstractname}

%TODO: English abstract

\selectlanguage{ngerman}
\section*{\centering\abstractname}

%TODO: Deutsches abstract

%----------------------------TEXT-----------------------------------------------

%TODO: Kapitel 1
\chapter{Einleitung}

\section{Motivation}
\section{Gliederung}

%TODO: Kapitel 2
\chapter{Mobile Systeme}

\section{Aktuelle Systeme auf dem mobilen Markt}
\subsection{iOS}

%TODO: Quellen
\subsection{Android}
Android ist ein Open Source Betriebssystem und gleichzeitig eine Software-Plattform, welches stark im mobilen Bereich vertreten ist und auf dem Linux-Kernel basiert. Zu finden ist diese auf Smartphones, Tablet-Computern, Netbooks und auch auf Smart-TV Geräten. Entwickelt wird Android von der Open Handset Alliance (OHA), welche von Google gegründet wurde. Die OHA wurde im November 2007 gegründet und ist ein Konsortium von mehr als 80 Unternehmen aus den Bereichen Mobilfunknetz, Geräteherstellung, Halbleiterindustrie, Marketing und Software. Der Grund für die Entwicklung von Android war und ist es, einen offenen Standard für mobile Geräte zu schaffen.

Durch seine Offenheit ermöglicht Android Entwicklern große Freiheit bei der Programmierung von Applikationen. Eigene Entwicklungen können auch mit Anwendungen von Google, wie zum Beispiel Google Maps, verknüpft werden.
Auch der Hardwarebereich bietet ein breites Spektrum an Geräten mit kostengünstigen, bis hochpreisigen Angeboten, sowohl mit einfacher bis qualitativ hochwertiger, technischer Ausstattung. Benutzer haben die Möglichkeit, ihre Geräte weitestgehend frei zu gestalten und einzustellen. Bei der Installation von Anwendungen sind sie auch nicht zwangsläufig an einen Anbieter gebunden.
Android Versionen sind nach süßen Leckereien benannt und dem Anfangsbuchstaben nach alphabetisch sortiert.
Android Versionen im Überblick:
%TODO: Versionsübersicht einfügen
%TODO: Punktaufzählung

Vorteile:
Open Source
Unabhängigkeit von Anbietern
Personalisierung
Hardwareangebot

Nachteile:
Hohe Verbreitung von Schadsoftware
Aktualität der Version ist abhängig vom Gerätehersteller

Quellen:
%http://www.openhandsetalliance.com/android_overview.html
%http://www.openhandsetalliance.com/oha_overview.html
%http://www.openhandsetalliance.com/oha_members.html
%https://source.android.com/source/build-numbers.html

\subsection{Windows Phone}
\subsection{Weitere Systeme}

\section{Bedarfsanalyse}
\subsection{Markt- und Useranteile der jeweiligen Systeme}
\subsection{Verfügbare Applikationen / Games der Stores}

%TODO: Kapitel 3
\chapter{Native Softwareentwicklung in den jeweiligen Systemen}
\section{Hardwarevorraussetzungen}
\section{Programmiersprachen}
\section{Entwicklungsumgebungen}

%TODO: Kapitel 4
\chapter{Cross-Plattform Entwicklung}
\section{Sinn und Gedanke von Cross-Plattform Entwicklung}
\section{Funktionsweise von Cross-Plattform Entwicklung}
\subsection{Technik}
\subsection{Geteilter Content}
\subsection{Übersetzung in die jeweiligen System}

%TODO: Kapitel 5
\chapter{Cross-Plattform Frameworks}
\section{Tools und Anbieter zur Entwicklung}
\section{Verweis auf Bachelorarbeit: „Plattformabhängige und –unabhängige Entwicklung mobiler Anwendungen am Beispiel von Geo-Wikipedia-App“}
\section{Gamespezifische Frameworks und Engines}
\subsection{Monogame}
\subsection{Cocos2D}
\subsection{Libgdx}
\subsection{Unity}
\subsection{Unreal Engine}
\subsection{Weitere Frameworks}
\section{Entwicklungsumgebungen}
\subsection{Unterstützte IDEs}
\subsection{Systembedingte Einschränkungen}

%TODO: Kapitel 6
\chapter{Gegenüberstellung der Frameworks}
%Features und Einschränkungen
\section{Zielplattformen}
\section{Skalierbarkeit der Menge der Plattformen} % ???
\section{Programmiersprachen}
\section{Unterstützung von 2D und 3D}
\section{Zugriff auf Hardware}%Accelerometer etc
\section{Free- und Pro- Versionen}
\section{Einfluss auf Einstellungen}
\section{Zusätzlich benötigte Software}
\section{Aktualität - Versionen - Community}
\section{Zukunftsaussichten}

%OPTIONAL!!!
%TODO: Kapitel 7
\chapter{Analyse der Marktanteile}
\section{Menge an Firmen und Entwicklern}
\section{Menge an Games}

%OPTIONAL!!!
%TODO: Kapitel 8
\chapter{Stores für mobile Spiele}
\section{Allgemeine Bedingungen für Entwickler}
\section{Maximale App-Größe}
\section{Anforderungen an den Quellcode}
\section{Kosten und Abgaben}

%TODO: Kapitel 9
\chapter{Kosten-Nutzen Vergleich}
%Vergleichstabelle zu den Variationen der Projektanforderungen im Zusammenhang mit den Möglichkeiten und Kosten der Frameworks

%TODO: Kapitel 10
\chapter{Grundgerüst und Aufbau eines Cross-Plattform Projekts}
\section{Geteilter Content}
\section{Plattformabhängiger Content}
\section{Grundaufbau bei Engines}
\section{Grundaufbau bei Frameworks}

%OPTIONAL!!!
%TODO: Kapitel 11
\chapter{Game-typische Design Patterns und Architekturen}
\section{Architekturen}
\section{Patterns}
\section{Verweis auf Bachelorarbeit: „Use of Design Patterns for mobile game Development“}

%OPTIONAL!!!
%TODO: Kapitel 12
\chapter{Nutzen von Architekturen}
\section{Pro}
\subsection{Skalierbarkeit}
\subsection{Lesbarkeit}
\subsection{Wiederverwertbarkeit}
\section{Contra}
\subsection{KISS - Keep it simple stupid}

%TODO: Kapitel 13
\chapter{Konzeption einer Applikation}
\section{Ideen}
\section{Anforderungen}
\section{User Stories}

%TODO: Kapitel 14
\chapter{Implementierung der Applikation}
\section{Verwendete Frameworks und Engines}
\section{Verwendete APIs und SDKs}
\section{Assets und deren Verwendung}

\chapter{Analyse messbarer Metriken}
%Kosten, Performance, Akkuverbrauch, App-Daten-Aufteilung, Größe des benötigten Speichers, Größe des geteilten Contents, Stabilität, RAM, Code-Zeilen/-Größe, Entwicklungszeit(bedingt zu berücksichtigen)

\chapter{Vergleich der Messprotokolle}

\chapter{Fazit}



%--------------------- VERZEICHNISSE -------------------------------------

\listoffigures % Abbildungsverzeichnis erzeugen
%\listoftables % Tabellenverzeichnis erzeugen

%------------------------------ LITERATURVERZEICHNIS----------------------
\begin{thebibliography}{}
\end{thebibliography}
%----------------------------- EIGENSTÄNDIGKEITSERKLÄRUNG-----------------
\clearpage\thispagestyle{empty}
\eigen  % im header definiert
%----------------------------- ENDE --------------------------------------
\end{document}